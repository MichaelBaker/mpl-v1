\documentclass[11pt]{article}
\usepackage{prftree}
\usepackage{turnstile}

% ----------------------------------------------
% -- Aliases

\newcommand{\subtype}{\leq}
\newcommand{\infers}{\Rightarrow}
\newcommand{\checks}{\Leftarrow}
\newcommand{\context}{\vdash}

% ----------------------------------------------
% -- Main Document

\begin{document}

\section{Inference Rules}

$\Gamma \vdash a \infers t$ means that under the assumptions $\Gamma$ we can infer that term $a$ has type $t$. These rules have $\Gamma$ and $a$ available as inputs and product $t$ as an ouput.

$$
\prfbyaxiom
	{infer integer literal}
	{\context 123 \infers \textnormal{integer}}
$$

$$
\prfbyaxiom
	{infer known symbol}
	{x : t \context x \infers t}
$$

$$
\prftree[r]
	{infer type annotation}
	{\context x \checks t}
	{\context x : t \infers t}
$$

$$
\prftree[r]
	{infer function body}
	{x:t \context b \infers s}
	{\context \#(x : t = b) \infers t \rightarrow s}
$$

$$
\prftree[r]
	{infer application}
	{\context f \infers a \rightarrow b}
	{\context x \infers c}
	{\context c \subtype a}
	{\context f \ x \infers b}
$$

\section{Checking Rules}

$\Gamma \context a \checks t$ means that under the assumptions $\Gamma$, we can prove that term $a$ has type $t$. These rules take $\Gamma$, $a$, and $t$ as inputs and produce a boolean output. Either $a$ has assumed type $t$ or it doesn't.

$$
\prftree[r]
	{check}
	{\context x \infers s}
	{s \subtype t}
	{\context x \checks t}
$$

\section{Subtype Rules}

$a \subtype b$ means that a type $a$ satisfies the constraints of a type $b$.

$$
\prfbyaxiom
	{consistent integers}
	{integer \subtype integer}
$$

\end{document}
